Software architecture

Frontend

For Front-end the a group member did research in multiple technologies for front-end. Then in discussions we voted for Angular for front-end. For more research see the research which can be found in the 30analysis\frontend_frameworks. There is also a page breakdown diagram and a page navigation diagram if one needs more information found in 100documentation\frontend.

Backend

As to lower the cost it is decided to use open source programming languages. The deployment environment supports open source as well. In this light it is chosen that Java is used for back-end development. The group is well experienced with Java which made a big impact on the choice as well. see the backend research for more information found in 30analysis\java_frameworks_backend.

Furthermore a research was done on which frameworks to use in this project. From this research it was decided to use JDBC because complex query's are required. Which are not possible with for example Hibernate. Furthermore it was decided that Jersey was to be used for the endpoints. The group preferred small frameworks over big frameworks which can do more then one wants.

Database

When we started our project we received a PostgreSQL database. Because in the groups experience this database is good and well equipped we decided continuing using PostgreSQL

Architectural decisions

One of the big decisions made is that the littlest amount of endpoints should be called as possible. The consequence of this is that we have few endpoints but many interfaces (object definitions). The communication between frontend and backend is thus less error prone as result. If one is to look through the front-end interface one will see that we almost exclusively use interfaces instead of classes. This is done because creating an object adhering to an interface takes has a lower memory footprint.

For more information about the endpoints see the Software architecture endpoints aesthetic.pdf document in the 100documentation directory.



