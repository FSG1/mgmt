% !TeX spellcheck = en_US

\chapter{Run with Docker}

For each part of the project there is a docker file which can be used to run the software.
The docker file automates the build process and encapsulates it into a container.
These containers runs on every operating system and does not need any external dependencies besides the installed docker daemon.
You can compose the separate containers to a full services which includes all parts of the project.

~\\
\textbf{This method is the recommended way to deploy the project to a server.}

\section{Database}
\label{docker:db}

\begin{lstlisting}[caption={Build Database Container}]
#!/bin/bash
cd database
docker build -t fmms-database .
\end{lstlisting}

\begin{lstlisting}[caption={Run Database Container}]
#!/bin/bash
docker run -d --name fmms-database -p 5432:5432 fmms-database
\end{lstlisting}


\section{Backend}
\label{docker:backend}

\begin{lstlisting}[caption={Build Backend Container}]
#!/bin/bash
cd backend
docker build -t fmms-backend .
\end{lstlisting}

\begin{lstlisting}[caption={Run Backend Container}]
#!/bin/bash
docker run -d --name fmms-backend -p 8080:8080 fmms-backend
\end{lstlisting}

\section{Frontend}
\label{docker:frontend}

\begin{lstlisting}[caption={Build Frontend Container}]
#!/bin/bash
cd frontend
docker build -t fmms-frontend .
\end{lstlisting}

\begin{lstlisting}[caption={Run Frontend Container}]
#!/bin/bash
docker run -d --name fmms-frontend -p 4200:4200 fmms-frontend
\end{lstlisting}



\section{Compose}
To run all parts of the software inside docker containers, Docker Compose can be used to run and supervise the docker containers.
Therefore a docker compose file is needed which defines the structure of the application and the needed parameters.
The following listing shows a docker compose file which contains all needed configuration to run the project on your local machine.
~\\
To use docker compose perform the following steps:
\begin{enumerate}
	\item Install Docker and Docker-Compose
	\item Build Database, Backend and Frontend as explained in sections \ref{docker:db}, \ref{docker:backend} and \ref{docker:frontend}
	\item Put the content of listing \ref{lst:compose} into a file named \glqq docker-compose.yml\grqq
	\item Run shell command \glqq docker-compose up -d\grqq{} in the directory with the file created in the previous step
\end{enumerate}


\begin{minipage}{\textwidth}
	\lstinputlisting[language={},caption={Docker Compose File},label={lst:compose}]{sources/compose.yml}
\end{minipage}
