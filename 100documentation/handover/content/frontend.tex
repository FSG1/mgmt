\chapter{Setup Frontend}
\label{ch:frontend}

\begin{enumerate}
	\item Clone the GitHub repository of the Frontend application \url{https://github.com/FSG1/frontend.git}
	
	\item Install the Angular CLI as it is described in the angular "Get Started" tutorial \url{https://angular.io/guide/quickstart}.
	
	\item Open the cloned frontend repository with a terminal and execute "npm install" to install all necessary node packages for the frontend application.
	
	\item Look at the "page breakdown diagramm.png" to understand the structure and the pages of the frontend. You can find it in the "mgmt" repository on GitHub \url{https://github.com/FSG1/mgmt/tree/master/100documentation/frontend}.
	
	\item Read the "Software architecture endpoints asthetic.pdf" to understand the connections to the backend application. You can find it also at the "mgmt" repository on GitHub \url{https://github.com/FSG1/mgmt/tree/master/100documentation}.
	
	\item To understand the source code read the documentation. \href{https://en.wikipedia.org/wiki/RTFM}{\textit{RTFM}}
	
	\item To start the application open the cloned frontend repository with a terminal and execute "ng serve". Point your browser to \url{http://localhost:4200} to view the frontend application.
	
	\item If you want to test the frontend application make yourself familiar with the Testing documentation of Angular, you can find it here \url{https://angular.io/guide/testing}. To start the test, open the cloned frontend repository with a terminal and execute "npm test". A browser window opens which shows you the results of your tests.
\end{enumerate}

