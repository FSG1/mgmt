\chapter{Setup Database}
\label{ch:database}

The database needs an already installed version of \href{https://www.postgresql.org/}{PostgreSQL}.
To import the database schema, first a new user and a new database have to be created.
After that, the SQL files from the \href{https://github.com/FSG1/database/tree/master/scripts}{"scripts"} directory can be imported to the database. Please take care of the order as it is indicated by the numbers.

Listing \ref{lst:database} shows how to create a user, a database and import all SQL files.
The script is written in bash and only works in Linux systems.
For Windows or MacOS please use a database client like \href{https://www.pgadmin.org/}{PgAdmin} or \href{https://www.jetbrains.com/datagrip/}{DataGrip}.

\textbf{Please be aware that the database repository does contain only the database schema. To import the data there is a SQL file inside the subversion repository which can be imported the same way as the other files.}

~\\
\begin{minipage}{\textwidth}
	\begin{lstlisting}[caption={Import database schema},label={lst:database}]
	# Create user "module"
	sudo -u postgres \
		psql --command "CREATE USER module WITH PASSWORD 'fmms';"
	
	# Create database "modulemanagement"
	createdb -O module modulemanagement
	
	for f in scripts/*.sql; do 
		sudo -u postgres psql -U module -d modulemanagement -f "$f";
	done
	\end{lstlisting}
\end{minipage}

