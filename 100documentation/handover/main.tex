\documentclass[a4paper,oneside,titlepage]{report}

\usepackage[english]{babel}
\usepackage[T1]{fontenc}
\usepackage[utf8]{inputenc}

\usepackage[pdftex]{graphicx} %%Grafiken in pdfLaTeX
\usepackage{a4wide} %%Kleinere Seitenränder = mehr Text pro Zeile.
\usepackage{longtable} %%Für Tabellen, die eine Seite überschreiten
\usepackage{pdflscape}
\usepackage{caption}
\usepackage{float}
\usepackage[nottoc]{tocbibind}
\usepackage{pdfpages}
\usepackage[rightcaption]{sidecap}
\usepackage[pdftex,scale={.8,.8}]{geometry}
\usepackage{layout}
\usepackage{subfigure}
\usepackage{setspace}
\usepackage{tabularx}

\usepackage[toc]{glossaries}
\usepackage[left,pagewise,modulo]{lineno}
\usepackage[pdftex,colorlinks=false,hidelinks,pdfstartview=FitV]{hyperref}

\usepackage{%
	array,
	booktabs,
	dcolumn,
	rotating,
	shortvrb,
	units,
	url,
	lastpage,
	longtable,
	lscape
}

\setlength{\parindent}{0pt}
\setlength{\parskip}{.5\baselineskip}

\usepackage{xcolor}
\usepackage{listings}
\lstset{basicstyle=\ttfamily,
	showstringspaces=false,
	commentstyle=\color{red},
	keywordstyle=\color{blue},
	frame=single,
	captionpos=b,
	numbers=left,
	language=bash
}

\usepackage{metainfo}
\usepackage[pagestyles,raggedright]{titlesec}

% % Wichtige Metainformationen über das Dokument

\def\Company{\textit{FSG1 - Fontys SoFa Group 1}}
\def\Institute{\textit{Fontys Venlo Techniek en Logistiek}}
\def\Course{\textit{Informatics}}

\def\BoldTitle{Documentation of FMMS Database Structure}
\def\Subtitle{Software Factory Group 1}
\def\Authors{Tobias Derksen}


\title{\textbf{\BoldTitle}\\\Subtitle}
\author{\Authors \\ \\ \\ \Institute\\ \Course}
\date{Venlo, \today}


\newenvironment{constraint}[1]{
	\newcommand{\Header}[1]{\textbf{##1}}
	\newcommand{\Row}[2]{\Header{##1} & ##2 \\ \hline}
	
	\newcommand{\Schema}[1]{\Row{Schema}{##1}}
	\newcommand{\Table}[1]{\Row{Table}{##1}}
	\newcommand{\Column}[1]{\Row{Column}{##1}}
	
	\newcommand{\Description}[1]{\Row{Description}{##1}}
	
	
	\newcommand{\NotNull}[1]{\Row{Schema}{##1}}
	
	\begin{table}[H]
		\centering
		\label{constraint:#1}
		\begin{tabular}{|p{0.3\textwidth}|p{0.7\textwidth}|}
			\hline
}{
		\end{tabular}
		\caption{#1}
	\end{table}
}

% Generiert Deckblatt automatisch
\AtBeginDocument{
	\maketitle
	\thispagestyle{empty}
}

% Generiert PDF Informationen
\hypersetup{pdfinfo={
		Title={\BoldTitle},
		Author={\Authors},
		Subject={\Subtitle}
	}}
	
\usepackage{etoolbox}
\patchcmd{\chapter}{plain}{short}{}{}


\newpagestyle{long}{%
	\sethead[\thepage][][\chaptername\ \thechapter:\ \chaptertitle]{\chaptername\ \thechapter:\ \chaptertitle}{}{\thepage}
	\headrule
}

\newpagestyle{short}{%
	\sethead[\thepage][][]{}{}{\thepage}
	\headrule
}


%%%%%%%%%%%%%%%%%%%%%%%%%%%%%%%%%%%%%%%%%%%%%%%%%%%%%%%%%%%%%
%% DOKUMENT
%%%%%%%%%%%%%%%%%%%%%%%%%%%%%%%%%%%%%%%%%%%%%%%%%%%%%%%%%%%%%
\begin{document}

\pagenumbering{roman}
\DeclareGraphicsExtensions{.pdf,.jpg,.png}

% Deckblatt wird automatisch generiert.
\setcounter{page}{2}
\pagestyle{short}

\newpage
\tableofcontents % Inhaltsverzeichnis


\newpage


\pagestyle{long}

\pagenumbering{arabic}
\cleardoublepage


% Den Inhalt des Dokuments einbinden
% !TeX spellcheck = en_US

\chapter{Introduction}
This document specifies all the requirements of the SOFA project during the seventh semester of the Software Engineering course at Fontys Hogeschool Techniek en Logistiek. The project is named \gls{FMMS}.

\section{Purpose}
The purpose of this project is to streamline the process of changing module descriptions for the classes in the Software Engineering course, offering one centralized source of truth for the information contained in the module descriptions and the \gls{OER}, and a controlled way of changing the descriptions if need be.

\section{Scope}
The scope of this document contains all the requirements of the system to be created.

\section{Overview}
This document is divided into three chapters. Chapter one will introduce the document and its purpose. The second chapter will give an overview of the functionality of the system and other interactions. The third chapter will specify all detailed requirements gathered which shall be implemented.

% !TeX spellcheck = en_US

\chapter{Software Architecture}




\chapter{Setup Frontend}


\begin{enumerate}
	\item Clone the GitHub repository of the Frontend application \url{https://github.com/FSG1/frontend.git}
	
	\item Install the node package manager (npm) on your system, if it‘s not installed. You can find the software here \url{https://www.npmjs.com/get-npm?utm\_source=house\&utm\_medium=homepage\&utm\_campaign=free\%20orgs\&utm\_term=Install\%20npm}
	
	\item Install the Angular CLI globally aith "npm install -g @angular/cli" in a terminal. Refer to the "Get Started" guide of Angular \url{https://angular.io/guide/quickstart} for more information.
	
	\item Open the cloned frontend repository with a terminal and execute "npm install" to install all necessary node packages for the frontend application.
	
	\item Look at the "page breakdown diagramm.png" to understand the structure and the pages of the frontend. You can find it in the "mgmt" repository on GitHub \url{https://github.com/FSG1/mgmt/tree/master/100documentation/frontend}.
	
	\item Read the "Software architecture endpoints asthetic.pdf" to understand the connections to the backend application. You can find it also at the "mgmt" repository on GitHub \url{https://github.com/FSG1/mgmt/tree/master/100documentation}.
	
	\item To understand the programmcode read the documenation of the programmcode.
	
	\item To start the application open the cloned frontend repository with a terminal and execute "ng serve". Point your browser to \url{http://localhost:4200} to view the frontend application.
	
	\item If you want to test the frontend application make yourself familiar with the Testing documentation of Angular, you can find it here \url{https://angular.io/guide/testing}. To start the test, open the cloned frontend repository with a terminal and execute "npm test". A browser window opens which shows you the results of your tests.
\end{enumerate}



\documentclass{report}
\usepackage{parskip}
\usepackage{listings}
\usepackage{hyperref}
\hypersetup{
	hidelinks = true
}
\begin{document}
	The backend of FMMS consists of a REST API connected to a PostgreSQL database. The REST API is written in Java using Jersey. Jersey is an open source framework that supports JAX-RS, which is a simple API spec for creating REST APIs. The Jersey documentation can be seen at \url{https://jersey.github.io/}.
	
	To checkout the project:
	\begin{enumerate}
		\item First install Maven if you do not have it already. \\ 
		\url{https://maven.apache.org/install.html}
		\item Clone the GitHub repository with \texttt{git clone git@github.com:FSG1/backend.git}
	\end{enumerate}
	
	The backend can be run in Docker or standalone from the jar.
	\begin{itemize}
		\item To build and run the backend in Docker, refer to the chapter on Docker deployment.
		\item To build and run the backend standalone, run \texttt{mvn build} followed by \texttt{mvn exec:java} in the root of the project, or use your IDEs built in build-and-run functionality.	
	\end{itemize}
\end{document}

% !TeX spellcheck = en_US

\chapter{Run with Docker}

For each part of the project there is a docker\footnote{\url{https://www.docker.com/get-docker}}  file which can be used to run the software.
The docker file automates the build process and encapsulates it into a container.
These containers runs on every operating system and does not need any external dependencies besides the installed docker daemon.
You can compose the separate containers to a full services which includes all parts of the project.

\section{Database}
\label{docker:db}

To setup a database server and create the proper users and databases can be very error-prone. 
Therefore a docker container can be used which automatically sets up the user, the database and tables.

During the creation process, the SQL files in the "scripts" directory are executed, sorted by name (that's why there are numbers in front of the filenames).
The database repository only contains the table structure and does not contain any data. Nevertheless, data can be automatically imported by copying a proper SQL file into the "scripts" directory before building the container.

When running the database as a docker container, please make sure that there are no other databases running on port 5432 or change the port mapping.

\begin{minipage}{\textwidth}
\begin{lstlisting}[caption={Build and run Database Container}]
cd database
# Build container
docker build -t fmms-database .
# Run container
docker run -d --name fmms-database -p 5432:5432 fmms-database
\end{lstlisting}
\end{minipage}

\section{Backend}
\label{docker:backend}

The configuration of the Backend API can be done without any changes to the source code.
During initialization, environment variables \footnote{\url{https://en.wikipedia.org/wiki/Environment_variable}} are read and the values will be used as configuration.
There are basically three important parts to configure:
\begin{itemize}
	\item The Backend URI containing port and base url
	\item The database connection
	\item Username and password for restricted actions
\end{itemize}

Restricted actions are all actions which can change the data in the database. The backend uses HTTP Basic authentication to authenticate users who wants to perform restricted actions. The credentials are currently hard-coded into the configuration and can be set via environment variables.

The default values has been chosen to allow the software to be run locally. For server deployment other values need to be entered.
The default database URL contains the default docker host ip address, which implies that a PostgreSQL server is bound to the port 5432 of the host.

The environment configuration can be given into the docker containers using the docker environment functionality (see the Docker documentation \footnote{\url{https://docs.docker.com/engine/reference/run/\#env-environment-variables}}). 

\begin{minipage}{\textwidth}
\begin{table}[H]
	\renewcommand{\arraystretch}{1.2}
	\centering
	\begin{tabularx}{\textwidth}{|l|l|X|}
		\hline
		\textbf{Name} & \textbf{Default Value} &  \\ \hline
		HOST & 0.0.0.0 & IP Address to bind server socket to. Usually the default value will do the job. \\ \hline
		PORT & 8080 & Server port to listen on \\ \hline
		BASE & /fmms & API Base URI \\ \hline
		DB & 172.17.0.1:5432/modulemanagement & DB URL for JDBC postgres driver format: <IP>:<PORT>/<databasename> \\ \hline
		DB\_USER & fmms & Username to access the database \\ \hline
		DB\_PASSWD & fmms & Password to access the database \\ \hline
		AUTH\_USER & fmms & Username for restricted actions \\ \hline
		AUTH\_PASSWORD & modulemanagement & Password for restricted actions \\ \hline
	\end{tabularx}
	\caption{Environment Configuration for Backend}
	\label{backend-env}
\end{table}
\end{minipage}

\begin{minipage}{\textwidth}
\begin{lstlisting}[caption={Build Backend Container}]
cd backend
# Build container
docker build -t fmms-backend .
# Run container
docker run -d --name fmms-backend -p 8080:8080 fmms-backend
\end{lstlisting}
\end{minipage}

\section{Frontend}
\label{docker:frontend}

The frontend configuration needs to be done inside the source code before building the container.
The default configuration works only for local use and is not suitable for server deployment.
The configuration is done via environment files which are loaded based on cli arguments. \footnote{\url{http://tattoocoder.com/angular-cli-using-the-environment-option/}}


\begin{minipage}{\textwidth}
\begin{lstlisting}[caption={Build Frontend Container}]h
cd frontend
# Build container
docker build -t fmms-frontend .
# Run container
docker run -d --name fmms-frontend -p 4200:4200 fmms-frontend
\end{lstlisting}
\end{minipage}




\section{Compose}
\label{docker:compose}
To run all parts of the software inside docker containers, Docker Compose\footnote{\url{https://docs.docker.com/compose/install/}} can be used to run and supervise the docker containers.
Therefore a docker compose file is needed which defines the structure of the application and the needed parameters.
The following listing shows a docker compose file which contains all needed configuration to run the project on your local machine.
~\\
To use docker compose perform the following steps: \\ \\
\begin{minipage}{\textwidth}
\begin{enumerate}
	\item Install Docker and Docker-Compose
	\item Build Database, Backend and Frontend as explained in sections \ref{docker:db}, \ref{docker:backend} and \ref{docker:frontend}
	\item Put the content of listing \ref{lst:compose} into a file named \glqq docker-compose.yml\grqq
	\item Run shell command \glqq docker-compose up -d\grqq{} in the directory with the file created in the previous step
\end{enumerate}
\end{minipage}

\begin{minipage}{\textwidth}
	\lstinputlisting[language={},caption={Docker Compose File},label={lst:compose}]{sources/compose.yml}
\end{minipage}


\section{Deployment on server}
\label{docker:deployment}






\end{document}
