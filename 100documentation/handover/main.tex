\documentclass[a4paper,oneside,titlepage]{report}

\usepackage[english]{babel}
\usepackage[T1]{fontenc}
\usepackage[utf8]{inputenc}

\usepackage[pdftex]{graphicx} %%Grafiken in pdfLaTeX
\usepackage{a4wide} %%Kleinere Seitenränder = mehr Text pro Zeile.
\usepackage{longtable} %%Für Tabellen, die eine Seite überschreiten
\usepackage{pdflscape}
\usepackage{caption}
\usepackage{float}
\usepackage[nottoc]{tocbibind}
\usepackage{pdfpages}
\usepackage[rightcaption]{sidecap}
\usepackage[pdftex,scale={.8,.8}]{geometry}
\usepackage{layout}
\usepackage{subfigure}
\usepackage{setspace}
\usepackage{tabularx}

\usepackage[toc]{glossaries}
\usepackage[left,pagewise,modulo]{lineno}
\usepackage[pdftex,colorlinks=false,hidelinks,pdfstartview=FitV]{hyperref}

\usepackage{%
	array,
	booktabs,
	dcolumn,
	rotating,
	shortvrb,
	units,
	url,
	lastpage,
	longtable,
	lscape
}

\setlength{\parindent}{0pt}
\setlength{\parskip}{.5\baselineskip}

\usepackage{xcolor}
\usepackage{listings}
\lstset{basicstyle=\ttfamily,
	showstringspaces=false,
	commentstyle=\color{red},
	keywordstyle=\color{blue},
	frame=single,
	captionpos=b,
	numbers=left,
	language=bash
}

\usepackage{metainfo}
\usepackage[pagestyles,raggedright]{titlesec}

% % Wichtige Metainformationen über das Dokument

\def\Company{\textit{FSG1 - Fontys SoFa Group 1}}
\def\Institute{\textit{Fontys Venlo Techniek en Logistiek}}
\def\Course{\textit{Informatics}}

\def\BoldTitle{Personal Development Plan}
\def\Subtitle{Software Factory Group 1}
\def\Authors{Nils Nieuwenhuis}


\title{\textbf{\BoldTitle}\\\Subtitle}
\author{\Authors \\ \\ \\ \Institute\\ \Course}
\date{Venlo, \today}


% Generiert Deckblatt automatisch
\AtBeginDocument{
	\maketitle
	\thispagestyle{empty}
}

% Generiert PDF Informationen
\hypersetup{pdfinfo={
		Title={\BoldTitle},
		Author={\Authors},
		Subject={\Subtitle}
	}}
	
\usepackage{etoolbox}
\patchcmd{\chapter}{plain}{short}{}{}


\newpagestyle{long}{%
	\sethead[\thepage][][\chaptername\ \thechapter:\ \chaptertitle]{\chaptername\ \thechapter:\ \chaptertitle}{}{\thepage}
	\headrule
}

\newpagestyle{short}{%
	\sethead[\thepage][][]{}{}{\thepage}
	\headrule
}


%%%%%%%%%%%%%%%%%%%%%%%%%%%%%%%%%%%%%%%%%%%%%%%%%%%%%%%%%%%%%
%% DOKUMENT
%%%%%%%%%%%%%%%%%%%%%%%%%%%%%%%%%%%%%%%%%%%%%%%%%%%%%%%%%%%%%
\begin{document}

\pagenumbering{roman}
\DeclareGraphicsExtensions{.pdf,.jpg,.png}

% Deckblatt wird automatisch generiert.
\setcounter{page}{2}
\pagestyle{short}

\newpage
\tableofcontents % Inhaltsverzeichnis


\newpage


\pagestyle{long}

\pagenumbering{arabic}
\cleardoublepage


% Den Inhalt des Dokuments einbinden
% !TeX spellcheck = en_US


% !TeX spellcheck = en_US

\chapter{Software Architecture}

\section{Frontend}

For Frontend the group member did research in multiple technologies for front-end. Then in discussions we voted for Angular for frontend. For more information see the research which can be found in the "30analysis/frontend\_frameworks". There is also a page breakdown diagram and a page navigation diagram if one needs more information found in "100documentation/frontend".

\section{Backend}

As to lower the cost it is decided to use open source programming languages. The deployment environment supports open source as well. In this light it is chosen that Java is used for back-end development. The group is well experienced with Java which made a big impact on the choice as well. see the backend research for more information found in "30analysis/java\_frameworks\_backend".

Furthermore a research was done on which frameworks to use in this project. From this research it was decided to use JDBC because complex query's are required. Which are not possible with for example Hibernate. Furthermore it was decided that Jersey was to be used for the endpoints. The group preferred small frameworks over big frameworks which can do more then one wants.

\section{Database}

When we started our project we received a PostgreSQL database. Because in the groups experience this database is good and well equipped we decided continuing using PostgreSQL


\section{Architectural decisions}

One of the big decisions made is that the littlest amount of endpoints should be called as possible. The consequence of this is that we have few endpoints but many interfaces (object definitions). The communication between frontend and backend is thus less error prone as result. If one is to look through the front-end interface one will see that we almost exclusively use interfaces instead of classes. This is done because creating an object adhering to an interface takes has a lower memory footprint.

For more information about the endpoints see the Software "architecture endpoints aesthetic.pdf" document in the 100documentation directory.

% !TeX spellcheck = en_US

\chapter{Build}




% !TeX spellcheck = en_US

\chapter{Run}




% !TeX spellcheck = en_US

\chapter{Run with Docker}

For each part of the project there is a docker\footnote{\url{https://www.docker.com/get-docker}}  file which can be used to run the software.
The docker file automates the build process and encapsulates it into a container.
These containers runs on every operating system and does not need any external dependencies besides the installed docker daemon.
You can compose the separate containers to a full services which includes all parts of the project.

\section{Database}
\label{docker:db}

To setup a database server and create the proper users and databases can be very error-prone. 
Therefore a docker container can be used which automatically sets up the user, the database and tables.

During the creation process, the SQL files in the "scripts" directory are executed, sorted by name (that's why there are numbers in front of the filenames).
The database repository only contains the table structure and does not contain any data. Nevertheless, data can be automatically imported by copying a proper SQL file into the "scripts" directory before building the container.

When running the database as a docker container, please make sure that there are no other databases running on port 5432 or change the port mapping.

\begin{minipage}{\textwidth}
\begin{lstlisting}[caption={Build and run Database Container}]
cd database
# Build container
docker build -t fmms-database .
# Run container
docker run -d --name fmms-database -p 5432:5432 fmms-database
\end{lstlisting}
\end{minipage}

\section{Backend}
\label{docker:backend}

The configuration of the Backend API can be done without any changes to the source code.
During initialization, environment variables \footnote{\url{https://en.wikipedia.org/wiki/Environment_variable}} are read and the values will be used as configuration.
There are basically three important parts to configure:
\begin{itemize}
	\item The Backend URI containing port and base url
	\item The database connection
	\item Username and password for restricted actions
\end{itemize}

Restricted actions are all actions which can change the data in the database. The backend uses HTTP Basic authentication to authenticate users who wants to perform restricted actions. The credentials are currently hard-coded into the configuration and can be set via environment variables.

The default values has been chosen to allow the software to be run locally. For server deployment other values need to be entered.
The default database URL contains the default docker host ip address, which implies that a PostgreSQL server is bound to the port 5432 of the host.

The environment configuration can be given into the docker containers using the docker environment functionality (see the Docker documentation \footnote{\url{https://docs.docker.com/engine/reference/run/\#env-environment-variables}}). 

\begin{minipage}{\textwidth}
\begin{table}[H]
	\renewcommand{\arraystretch}{1.2}
	\centering
	\begin{tabularx}{\textwidth}{|l|l|X|}
		\hline
		\textbf{Name} & \textbf{Default Value} &  \\ \hline
		HOST & 0.0.0.0 & IP Address to bind server socket to. Usually the default value will do the job. \\ \hline
		PORT & 8080 & Server port to listen on \\ \hline
		BASE & /fmms & API Base URI \\ \hline
		DB & 172.17.0.1:5432/modulemanagement & DB URL for JDBC postgres driver format: <IP>:<PORT>/<databasename> \\ \hline
		DB\_USER & fmms & Username to access the database \\ \hline
		DB\_PASSWD & fmms & Password to access the database \\ \hline
		AUTH\_USER & fmms & Username for restricted actions \\ \hline
		AUTH\_PASSWORD & modulemanagement & Password for restricted actions \\ \hline
	\end{tabularx}
	\caption{Environment Configuration for Backend}
	\label{backend-env}
\end{table}
\end{minipage}

\begin{minipage}{\textwidth}
\begin{lstlisting}[caption={Build Backend Container}]
cd backend
# Build container
docker build -t fmms-backend .
# Run container
docker run -d --name fmms-backend -p 8080:8080 fmms-backend
\end{lstlisting}
\end{minipage}

\section{Frontend}
\label{docker:frontend}

The frontend configuration needs to be done inside the source code before building the container.
The default configuration works only for local use and is not suitable for server deployment.
The configuration is done via environment files which are loaded based on cli arguments. \footnote{\url{http://tattoocoder.com/angular-cli-using-the-environment-option/}}


\begin{minipage}{\textwidth}
\begin{lstlisting}[caption={Build Frontend Container}]h
cd frontend
# Build container
docker build -t fmms-frontend .
# Run container
docker run -d --name fmms-frontend -p 4200:4200 fmms-frontend
\end{lstlisting}
\end{minipage}




\section{Compose}
\label{docker:compose}
To run all parts of the software inside docker containers, Docker Compose\footnote{\url{https://docs.docker.com/compose/install/}} can be used to run and supervise the docker containers.
Therefore a docker compose file is needed which defines the structure of the application and the needed parameters.
The following listing shows a docker compose file which contains all needed configuration to run the project on your local machine.
~\\
To use docker compose perform the following steps: \\ \\
\begin{minipage}{\textwidth}
\begin{enumerate}
	\item Install Docker and Docker-Compose
	\item Build Database, Backend and Frontend as explained in sections \ref{docker:db}, \ref{docker:backend} and \ref{docker:frontend}
	\item Put the content of listing \ref{lst:compose} into a file named \glqq docker-compose.yml\grqq
	\item Run shell command \glqq docker-compose up -d\grqq{} in the directory with the file created in the previous step
\end{enumerate}
\end{minipage}

\begin{minipage}{\textwidth}
	\lstinputlisting[language={},caption={Docker Compose File},label={lst:compose}]{sources/compose.yml}
\end{minipage}


\section{Deployment on server}
\label{docker:deployment}






\end{document}
