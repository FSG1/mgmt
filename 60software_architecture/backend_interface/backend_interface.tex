\documentclass{article}
\usepackage[usenames,dvipsnames]{xcolor}
\usepackage{tcolorbox}
\usepackage{tabularx}
\usepackage{array}
\usepackage{colortbl}
\tcbuselibrary{skins}

\newcolumntype{Y}{>{\raggedleft\arraybackslash}X}

\tcbset{tab1/.style={fonttitle=\bfseries\large,fontupper=\normalsize\sffamily,
		colback=yellow!10!white,colframe=red!75!black,colbacktitle=Salmon!40!white,
		coltitle=black,center title,freelance,frame code={
			\foreach \n in {north east,north west,south east,south west}
			{\path [fill=red!75!black] (interior.\n) circle (3mm); };},}}

\tcbset{tab2/.style={enhanced,fonttitle=\bfseries,fontupper=\normalsize\sffamily,
		colback=yellow!10!white,colframe=red!50!black,colbacktitle=Salmon!40!white,
		coltitle=black,center title}}

\begin{document}
	
	\section{Interface specification}
	This document outlines the specifications of the back-end application. The back-end application has a RESTful interface with which the front-end communicates with. \\ \\
	The data format used will be JSON. \\ \\
	The JSON objects can be found in the json\_objects.ods file in this directory.
	
	\begin{tcolorbox}[tab2,tabularx={X||Y|Y|Y|Y||Y},title=Complete overview of back-end endpoints. P stands for parameter,boxrule=0.5pt]
		Uri path without source  & Description    \\\hline\hline
		/curricula  & an array of curricula  \\\hline
		/modules/:module\_code  & returns a module.  \\\hline
		/curriculum/:curriculum\_id/semesters  & an array of semesters 
	\end{tcolorbox}
	
	
	\section{Sprint implementation}
	
	For the second sprint the /modules/:module\_code will be implemented. example would be /modules/2

	\begin{tcolorbox}[tab2,tabularx={X||Y|Y|Y|Y||Y},title=returns a module,boxrule=1pt]
	URL & /modules/:module\_code    \\\hline
	Method   & GET \\\hline
	Returns &  Returns a filled module\_content object as specified in sprint\_two\_JSON\_objects.png \\\hline
	Returns & 200 OK \\ & 404 Not found
	\end{tcolorbox}

	
	\section{Endpoint definitions}
	
	This section describes the urls more specifically with functionality, the JSON format and HTTP codes. The dataobjects are described in json\_objects.pdf in this repository\\\\
	
	\subsection{Curricula of student program}

	\begin{tcolorbox}[tab2,tabularx={X||Y|Y|Y|Y||Y},title=semesters of curriculum,boxrule=1pt]
		URL & /curriculum/:curriculum\_id/semesters   \\\hline
		Method   & GET \\\hline
		Returns &  json array of semester objects which in turn are filled with corresponding modules. \\\hline
		Returns & 200 OK \\ & 404 Not found  
	\end{tcolorbox}
	

	
	\begin{tcolorbox}[tab2,tabularx={X||Y|Y|Y|Y||Y},title=curricula of student program,boxrule=1pt]
		URL & /curricula    \\\hline
		Method   & GET \\\hline
		Returns &  json array of curriculum objects. \\\hline
		Returns & 200 OK \\ & 404 Not found 
	\end{tcolorbox}



\end{document}


















