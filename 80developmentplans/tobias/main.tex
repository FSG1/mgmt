% !TeX spellcheck = en_US
\documentclass[a4paper,oneside,titlepage]{article}

\usepackage[english]{babel}
\usepackage[T1]{fontenc}
\usepackage[utf8]{inputenc}

\usepackage[bottom]{footmisc}
\usepackage{a4wide} %%Kleinere Seitenränder = mehr Text pro Zeile.
\usepackage{pdflscape}
\usepackage{caption}
\usepackage{float}
\usepackage{pdfpages}
\usepackage[toc,page]{appendix}
\usepackage[pdftex,scale={.8,.8}]{geometry}
\usepackage{layout}
\usepackage[doublespacing]{setspace}
\usepackage[pdftex,colorlinks=false,hidelinks,pdfstartview=FitV]{hyperref}

\usepackage{%
	array,
	booktabs,
	dcolumn,
	rotating,
	shortvrb,
	url,
	longtable,
	lscape,
	tabularx
}

\setlength{\parindent}{0pt}
\setlength{\parskip}{.5\baselineskip}

\usepackage{metainfo}
\usepackage[pagestyles,raggedright]{titlesec}
\usepackage{xcolor,colortbl}

% % Wichtige Metainformationen über das Dokument

\def\Company{\textit{FSG1 - Fontys SoFa Group 1}}
\def\Institute{\textit{Fontys Venlo Techniek en Logistiek}}
\def\Course{\textit{Informatics}}

\def\BoldTitle{Personal Development Plan}
\def\Subtitle{Software Factory Group 1}
\def\Authors{Nils Nieuwenhuis}


\title{\textbf{\BoldTitle}\\\Subtitle}
\author{\Authors \\ \\ \\ \Institute\\ \Course}
\date{Venlo, \today}


% Generiert Deckblatt automatisch
\AtBeginDocument{
	\maketitle
	\thispagestyle{empty}
}

% Generiert PDF Informationen
\hypersetup{pdfinfo={
	Title={\BoldTitle},
	Author={\Authors},
	Subject={\Subtitle}
}}


\begin{document}

\DeclareGraphicsExtensions{.pdf,.jpg,.png}

\setcounter{page}{2}

\clearpage

\section{Introduction}
\label{sec:intro}

% show learning goals
% Explain which part will contribute to which learning goal
% Explain how document is structured. 

This document defines the development of personal skills during the Software Factory (SoFa) module.
During this module some learning goals should be met which contributes to the final skills of the Informatics study.
~\\
Section \ref{sec:current} provides an overview over the current skill level (see table \ref{sec:current}) according to the HBO-I matrix as defined in \glqq HBO-I Domain Description Bachelor of ICT\grqq.

The next section deals with the target skill level (see table \ref{sec:target}) and how this level will be accomplished. The concrete improvements are marked in red.

The third section explains how the target skills will be reached in detail. It defines how the progress will be measured and when at which point the target will be reached.

As part of the learning goals, a topic has to researched. This research topic will be described in Section \ref{sec:research}.



\subsection{Learning Goals}
\label{ssec:learning}
The learning goals of the module Software Factory (SoFa) as defined in the module description are:

\begin{itemize}
	\item[\textbf{LG1}] Show professional behavior in a project with a real customer (e.g. communication, collaboration, problem orientation and effectiveness, criteria-based decision making, systematic and well-structured process)
	\item[\textbf{LG2}] Fulfill a function relevant for a development project (e.g. project manager, scrum master, quality manager, configuration manager, software architect, software developer – other functions need to be approved the group’s coach)
	\item[\textbf{LG3}] Define a research topic relevant to the project, do the research, report on it and care foradequate application of the results in the project.
	\item[\textbf{LG4}] Deliver a relevant contribution to the project (next to his/her function and research topic) in three activities (of Manage, Analyse, Advice, Design and Realise) on any architectural layer (User Interaction, Business Processes, Infrastructure, Software, Hardware Interfacing)
	\item[\textbf{LG5}] Work together (communication, systematic and well-structured process).
	\item[\textbf{LG6}] Deliver results relevant to the customer and of adequate quality.
\end{itemize}
\captionof{table}{Learning Goals according to the Module Description}
\label{table:learninggoals}



% Introduction


% !TeX spellcheck = en_US

\section{Current Skill Level}
\label{sec:current}

The following table describes the current skill level. 

\begin{table}[H]
	\centering
	\begin{tabular}{|c|c|c|c|c|c|c|c|}
		\hline
		& Manage & Analyse & Advice & Design & Implement & \vtop{\hbox{\strut Professional}\hbox{\strut Behaviour}} & Research Skills \\ \hline
		User-Interface & & & & & & & \\ \hline
		Business Processes & & & & & & & \\ \hline
		Infrastructure & 2 & 2 & & 2 & 2 & & \\ \hline
		Software & 2 & 2 & 3 & 3 & 2 & & \\ \hline
		Hardware Interfacing & & & & & & & \\ \hline
		Professional Skills & & & & & & 2 & 2 \\ \hline
	\end{tabular}
	\caption{Current Skills Level}
	\label{currentskills}
\end{table}

% current skills level


% !TeX spellcheck = en_GB

\section{Target Skill Level}
\label{sec:target}


\begin{table}[H]
	\centering
	\begin{tabular}{|c|c|c|c|c|c|c|c|}
		\hline
		& Manage & Analyse & Advice & Design & Implement & Professional Behaviour & Research Skills \\ \hline
		User-Interface & & & & & & & \\ \hline
		Business Processes & & & & & & & \\ \hline
		Infrastructure &  &  & &  &  & & \\ \hline
		Software & &  &  &  &  & & \\ \hline
		Hardware Interfacing & & & & & & & \\ \hline
		Professional Skills & & & & & &  \\ \hline             
	\end{tabular}
	\caption{Target Skill Level}
	\label{targetskills}
\end{table}
% target skill level


\section{Development}
The developments of my competences will be described here.

Software Manage 2 to Software Manage 2:
During the SoFa project I will set up change and release management processes for the project I'm working on.

Software Analyse 2 to Software Analyse 3:
During the SoFa project I will participate in building the Software Requirements Specification document and in forming acceptance criteria for the user stories for each sprint.

Software Design 2 to Software Design 3:
During the SoFa project I will participate in making test strategies for our tests and in designing the system according to requirements and quality aspects.
% development plan


\section{Research Topic}
\label{sec:research}

To be defined. \\ \\
Can be researching in continuous integrations and deployment. \\
Can be research in test strategies or integration strategies. \\
Can be research in cloud providers.
% research topic


% Evaluation

\end{document}
