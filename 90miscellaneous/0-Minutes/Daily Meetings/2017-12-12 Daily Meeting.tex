% !TeX spellcheck = en_US
\documentclass[11pt]{meetingmins}


\usepackage{float}


\setcommittee{Fontys Venlo SoFa Group 1}

\setmembers{
	N.~Nieuwenhuis,
	L.~Ehren,
	S.~Brauer,
	T.~Derksen
}

\newcommand{\allpresentLoek}{
\setpresent{
	\chair{L.~Ehren},
	N.~Nieuwenhuis,
	S.~Brauer,
	\secretary{T.~Derksen}
}
}

\newcommand{\allpresentNils}{
	\setpresent{
		\chair{N.~Nieuwenhuis},
		L.~Ehren,
		S.~Brauer,
		\secretary{T.~Derksen}
	}
}

\newenvironment{results}[0]{
	\newcommand{\entry}[2]{\textbf{##1} & ##2 \\ \hline}
	\newcommand{\sjoerd}[1]{\entry{Sjoerd}{##1}}
	\newcommand{\nils}[1]{\entry{Nils}{##1}}
	\newcommand{\loek}[1]{\entry{Loek}{##1}}
	\newcommand{\tobias}[1]{\entry{Tobias}{##1}}
	
	\begin{table}[H]
		\centering
		\begin{tabular}{|p{0.2\textwidth}|p{0.8\textwidth}|}
			\hline
}{
		\end{tabular}
	\end{table}
}

\setdate{December 12, 2017}
\allpresentLoek

\begin{document}
\maketitle

\section{Agenda}
\begin{hiddenitems}
	\item What did we do last time
	\item What will we do today
	\item Discussion about PDF generation
	\item Customer discussion about PDF generation
\end{hiddenitems}

\section{What did we do last time}
\begin{results}
	\sjoerd{Researched possible solution to generate PDF files for module descriptions; Improved module edit component and updated software architecture}
	\nils{Worked in edit learning goals and edit exam information}
	\loek{Finished endpoints for editing a module}
	\tobias{Created template for the poster (for MON); Research different options to generate PDFs for module description}
\end{results}

\vspace{1em}

\section{What will we do today}
\begin{results}
	\sjoerd{Work on frontend aesthetics; Discuss solutions for PDF generation problem.}
	\nils{Finish and test edit learning goals and edit exam information parts of frontend}
	\loek{Write documentation on how to use the project; Work on personal development plan}
	\tobias{Work on personal research; Discuss solutions to the PDF generation problem}
\end{results}


\section{Discussion about PDF generation}
There are different options for creating PDF inside the browser and download it to the client:
\begin{itemize}
	\item JS PDF
	\item Pdf Make
	\item PhantomJS
	\item Browser's native print support
\end{itemize}

All of them have several serious disadvantages.
JS PDF and PdfMake can either print an image of the shown module description into a pdf or we have to create the whole description manually by calculating all positions.
~\\
PhantomJS can render any HTML including Javascript to different output types. But we need to create a proper HTML with CSS styling to get this running properly. 
~\\
The browser's native print support is highly dependent on the concrete browser and the operating system which is running. It produces different output on different browsers and platforms.

\section{Customer discussion about PDF generation}
The option we research and all the disadvantages we found (as described above) has been presented to our customer.
The customer was not happy which one of the given options.
We discussed using LaTex to generate a proper PDF. 
Fontys does not yet have a proper LaTex template for module descriptions, the descriptions are currently managed in a document format.
Therefore we need to create a proper LaTex template first, which we then can fill with concrete data and give to a proper compiler.
~\\
We discussed, that the time remaining for this module is limited and we probably are not able to create a full LaTex template. We will now divide the module description into parts.
We will implement these parts one after each other, so we get a working generation at the end of the sprint, but possibly not all parts of the module description are in there. 

\end{document}
