Project existence description

At every university and every study program of a university they do the same things differently. 
Every education in the Netherlands has a different approach at designing a curriculum and their corresponding modules.
All these curricula and modules are build up somewhat similar. So is it not better to create modules in a standardized way?
This is how the Fontys Module Management System (FMMS) project came to be.

The FMMS makes designing curricula and modules easier for those that need to design them. It visualizes all the important parts so it's easy to find relations within a curricula. The FMMS is intended to have Create, Read, Update and Delete (CRUD) functionality. It must be able to publish modules and Furthermore it must be able to take versioning into account where it tracks the changes.