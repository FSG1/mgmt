% !TeX spellcheck = en_US

\section{Sjoerd}
\label{sec:sjoerd}


\begin{goal}{Independently create good software}
	\Description{By being able to create software Independently my value for the team goes up as less time is wasted by asking others to inform me. Help can be asked if there's a risk of getting stuck and thus wasting more hours in total. But the focus lies on being able to create good quality software independently }
	\Measurement{Ask less and less help in the creation of software. measure points is: counting how many times I asked help in beginning, in mid and in the end}
	\Criteria{The times asked for help should be less in the end than in the beginning.}
	\Criteria{If the team finds my independently created code more useful then in the beginning.}
\end{goal}


\begin{goal}{Understandable software}
	\Description{The benefit of having understandable software is that software can be understood in less time. This is most valuable if other have to understand the written code. This can also save time if one has to read his own code after a year of absence from that code.}
	\Measurement{Group members who review my code can give feedback if its good or not. The amount of feedback measured is a measurement.}
	\Measurement{Understandability of my code and documentation.}
	\Criteria{If the amount of feedback is lower in the end in the beginning it's a success.}
	\Criteria{If my team finds the code and documentation more understandable. than in the beginning.}
\end{goal}

\begin{goal}{Improve software structuring}
	\Description{Having a good software structure can help finding certain code documents faster. Putting similar documents together creates a natural feel of where to find something and thus shorten searching time.}
	\Measurement{Measurement is the amount of feedback by fellow students. }
	\Measurement{Software structuring is clear to other students. }
	\Criteria{If the team finds my way of structuring more useful then in the beginning. }
\end{goal}