% !TeX spellcheck = en_US
\section{Introduction}
\label{sec:intro}


\subsection{Overview}

This document provides an overview over the database structure of the Fontys Module Management System (FMMS).
The target group of this document are informatics students of at least semester 7 or persons with a similar or better understanding of information technologies and databases.

The database management system (DBMS) which is used is PostgreSQL, which provides a full set up up-to-date database feature in an open source software product.
Initially the database was created by HVD but has been improved during the project.


\subsection{Conventions}

All database names follow a special convention which is described in this section.


\paragraph{Table Names}
All tables are named after the entity they represent. The name is always singular. If the table represents a many-to-many relationship, the table name contains the two tables which are connected separated by an underscores. The order of the tables is not regulated, but normally there is a natural direction inside the relationship.

\paragraph{Primary Keys}
All tables contain a column named "id" with the database type "SERIAL" which implies integer. This column is the primary key of each table. If there are business keys or natural unique combinations, they are created separately keeping "id" as the only primary key column to every table.

\paragraph{Foreign Keys}
All columns which contain a foreign key relationship are named using a special convention. They contain a regular name, which is mostly the table name they reference, followed by an underscore and then the column name of the foreign table which they reference. Therefore, all columns which contain an underscore represent a relationship to another table and hold a foreign key.

\paragraph{Data Types}
For each column the data type has been chosen which fits best to the intended data. There are two exceptions: first, all primary key columns have the data type "SERIAL" which implies an "INTEGER" (as described before), second all columns which contains character have the data type "TEXT". Because there is no performance or data storage difference in PostgreSQL database between "TEXT" and "CHARACTER VARYING" we do not need to keep track of max string length when we use "TEXT" by default.