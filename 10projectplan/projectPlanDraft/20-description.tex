\chapter{Description}
In this chapter is a description of the assignment for this project followed by an explanation 
of the problem which is to be solved. Finally the goals of the assignment will be set.

\section{Problem}
The customer wants a product to easily create, modify and publish module descriptions
for the Software Engineering and Business Informatics department of "Fontys Hogeschool Techniek en Logistiek" in the future.
\newline \\
Today there is no standardized way of creating and changing modules and their descriptions.
\newline \\
During the project our company will develop a software product to solve this problem.

\section{Assignment}
The assignment of this project is to create a system to manage module data within the Informatics
department of the Fontys University of Applied Science in Venlo. 

\newpage

\section{Goals}
\begin{itemize}
    \item Delivering a high quality product
    \item Minimum application displays the module data 
    \item Delivering proper documentation
    \item Delivering an extensible product
    \item Connection to a central database
\end{itemize}

\section{Approach}
To assure a high quality product with the minimum agreed functions we stick to the Agile development framework Scrum during the project.
A good documentation is achieved with our "SOFA Quality Guidelines" and "Software Requirements Specification" defined at the beginning of the project.
Careful consideration will have to be taking into account when designing parts of the system to ensure the end-product is extensible and easily understandable for possible other project teams who could continue with this project.

%\begin{itemize}
%    \item Scrum: To assure a high qualtiy product
%    \item Quality management & configuritaton management: .
%    \item Communication plan
%    \item Patterns
%\end{itemize}

