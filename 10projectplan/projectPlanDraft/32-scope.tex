\section{Project scope}
This section is about the scope of the project. It focuses on its functions and data, whereas the deliverables
are defined in section 3.3.1 as project products.
\newline \\
The target of the project is to create a system that makes the information collection and deployment of the module informations
easier. The most important feature of the system is to display the already available Information about modules and the consistency
of this information. Especially their learning goals connected to the hoger beroepsonderwijs.
    
    \subsection{Functionality}
    The functionality is described in the Software Requirements Specification document in detail,
    for more information please refer to this document.
    %The system shall provide the following functionalities:

    %\begin{itemize}
    %    \item A view to see the complete description of a module.
    %    \item Previous versions of the module descriptions.
    %    \item A release feature for the next semester, which also includes that changes during the semester aren't possible.
    %    \item A role system for users.
    %    \item A control system for changes with notifications.
    %    \item A Flow diagram to visualize the relationships between the modules.
    %\end{itemize}

    \subsection{Data}
    \begin{table}[H]
        \begin{tabular}{p{0.25\textwidth} p{0.5\textwidth}}
            \textbf{Module} & The name of the module \\
            \textbf{Semester} & The semster where the module takes place \\ 
            \textbf{Credits} & The credits which can be achieved \newline if the module is successfully completed \\
            \textbf{Valid as of} & The date were the descriptions will be valid \\
            \textbf{Author} & The creator of the description \\
            \textbf{Description} & The description of the module \\
            \textbf{Learning Goals} & The learning goals wich will be achieved during the module \\
            \textbf{Competence Profile} & The competence profile a student has after finishing the module \\
            \textbf{Module Assessment} & The assessmentforms of the module \\
            \textbf{Prior Knowledge} & The modules which have to be \newline completed to participate in the module \\
            \textbf{Additional Information} & Additional information to the module
        \end{tabular}
    \end{table}
    \pagebreak
