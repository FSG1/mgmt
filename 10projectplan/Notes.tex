Stakeholders

Van den Ham, Richard 	(customer)
IT Teachers 			(users)
Students 				(users of the program product)
Exam comission 			(users of the program product)
SOFA group				(Students who want to pass)
Jacobs, Jan				(Coach SOFA group) 



Stakeholders analysis.

Van den Ham, Richard
High power High interest. He is the customer and gains from having a succesful project.

IT Teachers 
High interest medium power, they gain from a succesful project. They might give input but are not directly involved.

Students
Medium interest low power, they profit from a good program because the teachers have more time for them. 

Exam Comission
Low interest low power, they profit from a good program.

SOFA group
high interest high power, Mandatory good project and product to pass the SOFA course.

Jacobs, Jan
high interest medium power, is assigned to the SOFA group and his goal is to see the SOFA group succeed.




Communication plan

Internal communication

Whatsapp for basic messages.
GIT for project documents and project.
GIT Everyone works on their own branch.
GIT main product is in main.
Weekly SCRUM meetings.
sprint meetings (frequency to be defined).
Daily standup meetings.


Stakeholder communication

Van den Ham, Richard. 
Weekly meeting.

Jacobs, Jan.
Weekly Meeting.



Success criteria

(product is what has been delivered, doesn't need to be finished)

1. The SOFA group passes the SOFA module.
2. The product passes the quality standard the team and customer agree upon.
3. The project should be easy to continue for a different (SOFA) team after this teams time runs up.
 





