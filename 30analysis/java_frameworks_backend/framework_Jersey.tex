\textbf{Jersey}\\
Jersey framework is more than the JAX-RS Reference Implementation. Jersey provides it's own API that extend the JAX-RS toolkit with additional features and utilities to further simplify RESTful service and client development. Jersey also exposes numerous extension SPIs so that developers may extend Jersey to best suit their needs.

Goals of Jersey project can be summarized in the following points:

Track the JAX-RS API and provide regular releases of production quality Reference Implementations that ships with GlassFish;
Provide APIs to extend Jersey \& Build a community of users and developers; and finally
Make it easy to build RESTful Web services utilising Java and the Java Virtual Machine.

Supports all criteria? \\
\begin{enumerate}
	\item Jersey can communicate with PostgreSQL but best in combination with another technology.
	\item Jersey supports HTTP. see source.
	\item Jersey has proper documentation see source.
	\item Jersey has tutorials see source.
\end{enumerate}

\textbf{Why use it?} \\
Jersey implements the JAX-RS api with extra features. Jersey makes it easier to create a restful service on any java application server. \\



\begin{enumerate}
	\item Fast and cutting edge
	\item Smooth JUnit integration
	\item Supports true asynchronous connections
\end{enumerate}
Cons:
\begin{enumerate}
	\item Jersey 2.0+ uses a somewhat messy dependency injection implemetation
	\item A large amount of online resources (3rd party) are related to Jersey 1.X making them unsuitable for Jersey 2.X
\end{enumerate}

\textbf{Sources}\\
https://jersey.github.io/
https://stackoverflow.com/questions/7052152/why-use-jax-rs-jersey
http://www.gajotres.net/best-available-java-restful-micro-frameworks/